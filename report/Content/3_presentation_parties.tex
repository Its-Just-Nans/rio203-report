\section{Présentation des parties}


\paragraph*{}
Pour créer notre solution de parking intelligent, nous avons opté pour une architecture assez simple et logique. Nous prévoyons au début d'utiliser un serveur centrale par parking mais cela nous imposait des limites. Nous avons donc revu l'architecture avec un serveur dans le cloud et les autres équipements dans le parking.

\paragraph*{}
Le parking possède donc des modules constitué d'un micro-contrôleur, d'une LED et d'un capteur. Chaque place de parking possède ce module capteur.
Le parking possède également à l'entrée un module de lecture de plaque d'immatriculation. Ce module est fait à l'aide d'un Rapsberry Pi et de deux caméras (entrée et sortie). Le raspberry Pi affiche également une interface Web pour l'utilisateur a l'entrée ou pour payer à la sortie.

\img{Content/images/image5.png}{Architecture du système global}{0.65}

\clearpage

\paragraph*{}
Afin de nous aider à nous organiser en amont du projet, nous avons réalisé un diagramme de Gantt. Cela nous a permis d'évaluer la durée de chaque tâche et de répartir le travail en fonction de la difficulté de la tâche et des aptitudes des étudiants.

\img{Content/images/gantt.png}{Diagramme de Gantt}{0.5}

\paragraph*{}
Lors de la réalisation, les tâches ont généralement duré une semaine de plus. Cela n'a néanmoins pas changé le rendu final car nous avons choisi de ne pas réaliser la fonction d'algorithme de direction - étant donné que nous avions fait cela en première année dans le cadre de l'UE \textbf{INF103} (java).

