\section*{Conclusion}
\addcontentsline{toc}{section}{Conclusion}

\paragraph*{}
La réalisation du projet de smart parking a été une expérience enrichissante pour notre équipe, marquant la conclusion de notre parcours pratique dans le domaine de l'IoT au sein du programme RIO. Tout au long de cette aventure, nous avons intégré des technologies telles que les capteurs ESP, un serveur en Node.js, et une Raspberry Pi pour l'analyse d'image avec OpenALPR, afin de créer un système efficace de gestion de parking.

L'implémentation de ces technologies a permis à certains membres du groupe de développer leurs compétences dans la conception et la mise en œuvre de solutions IoT concrètes. Les capteurs ESP ont été particulièrement utiles pour la collecte de données en temps réel, offrant une base solide pour la gestion intelligente des espaces de stationnement.

Le choix d'un serveur en Node.js a facilité la mise en place d'une architecture robuste et réactive, assurant une communication fluide entre les capteurs, la Raspberry Pi et d'autres composants du système. L'utilisation d'OpenALPR pour l'analyse d'image a permis d'automatiser la reconnaissance des plaques d'immatriculation, améliorant ainsi l'efficacité globale du système de smart parking.

Cependant, tout projet comporte des opportunités d'amélioration. Certains membres du groupe ont identifié des pistes d'optimisation telles que l'intégration d'une API bancaire pour une gestion plus avancée des transactions liées au stationnement. De plus, l'idée d'aller au-delà d'un simple proof of concept a été soulevée, suggérant la possibilité d'élargir les fonctionnalités du système pour répondre à des besoins plus complexes.

En ce qui concerne l'esthétique des interfaces, il a été noté que l'expérience utilisateur pourrait être encore améliorée pour rendre le système plus convivial. Des efforts supplémentaires pour affiner l'interface utilisateur contribueraient à renforcer l'acceptation et l'adoption du système par les utilisateurs finaux.

En conclusion, ce projet a été une étape significative dans notre parcours RIO, démontrant notre capacité à concevoir et mettre en \oe{}uvre des solutions IoT pratiques. Les leçons apprises, tant sur le plan technique que collaboratif, seront précieuses pour notre développement futur dans le domaine de l'IoT. Les opportunités d'amélioration identifiées offrent des perspectives passionnantes pour étendre et perfectionner notre système de smart parking, renforçant ainsi son utilité et son attrait sur le marché.

\section*{Liens}


\begin{itemize}
    \item \url{https://github.com/Its-Just-Nans/rio203} - Le web server
    \item \url{https://github.com/comeyrd/rio203-diagrams} - Diagrammes pour le rapport
    \item \url{https://github.com/katheleligaf/rio203-sensors} - Code des capteurs (ESP32, Rasperry Pico W)
    \item \url{https://github.com/comeyrd/rio203-image-detection} - Détection de plaque d'immatriculation
    \item \url{https://github.com/Its-Just-Nans/rio203-report} - Rapport
    \item \url{https://tinyurl.com/rio203} - Short link to this repo
\end{itemize}
